\documentclass[a4paper,10pt]{article}
\usepackage{graphicx}
\usepackage{amsmath}
\usepackage{hyperref}
\usepackage{geometry}
\geometry{margin=1in}
\usepackage{titlesec}
\usepackage{setspace}
\usepackage{natbib}
\renewcommand{\bibname}{References}
\bibliographystyle{apalike}
\renewcommand{\baselinestretch}{1.0}
\usepackage{helvet}
\renewcommand{\familydefault}{\sfdefault}
\usepackage[T1]{fontenc}

\title{Concept Paper: Reinforced Contrastive Knowledge Distillation For Mental Stress EGG Signal Processing}
\author{Kahunde Audrey(2300708649) & Amanyire Cindy(2300705985)}
\date{\today}

\begin{document}

\maketitle

\begin{abstract}
Stress is a significant health concern that affects cognitive and physiological functions. Traditional stress detection methods rely on subjective assessments, which lack reliability. This study explores stress detection using Electroencephalogram (EEG) signals and advanced machine learning techniques, including Knowledge Distillation, Reinforcement Learning, Sparse Learning, and Contrastive Learning. The objective is to develop an accurate, real-time stress detection model that can be integrated into wearable devices. By leveraging EEG signals, our model aims to enhance precision and efficiency while addressing issues such as computational complexity and model interpretability.\
\textbf{Keywords:} Mental Stress, EGG Signal Processing, Knowledge Distillation, Reinforcement Learning, Sparse Learning, Contrastive Learning.
\end{abstract}

\section{Background and Introduction}
Stress detection is crucial for mental health management. EEG signal processing has emerged as a promising approach due to its capability to capture real-time neural activity. 

\subsection{Technical Definitions}
\begin{itemize}
    \item \textbf{EEG Signal Processing}: Analyzing brain wave patterns to detect cognitive and emotional states.
    \item \textbf{Knowledge Distillation}: A technique to transfer knowledge from a large model to a smaller, efficient model.
    \item \textbf{Reinforcement Learning}: A method where an agent learns optimal actions by interacting with an environment.
    \item \textbf{Sparse Learning}: A technique focusing on selecting a subset of relevant features for learning.
    \item \textbf{Contrastive Learning}: A technique that enhances representations by distinguishing similar and dissimilar samples.
\end{itemize}

\subsection{Current Research Trends}
Recent research in stress detection emphasizes deep learning and physiological signal processing. Studies highlight EEG's effectiveness in recognizing stress patterns and the potential of machine learning for real-time stress monitoring.

\section{Problem Statement}

\subsection{Application Domain-Specific Problem}
Stress affects workplace productivity, learning outcomes, and overall well-being. Traditional detection methods are subjective, limiting real-time monitoring and intervention capabilities.

\subsection{Technical Computational Problem}
Challenges include:
\begin{itemize}
    \item High dimensionality of EEG signals.
    \item Model interpretability issues in deep learning.
    \item Sparse and noisy EEG data.
    \item Computational efficiency for real-time applications.
\end{itemize}

\section{Proposed Solution}
We propose an EEG-based stress detection model leveraging:
\begin{itemize}
    \item Knowledge Distillation to optimize deep learning models.
    \item Reinforcement Learning to improve adaptive stress classification.
    \item Sparse Learning to enhance feature selection and reduce computational costs.
    \item Contrastive Learning to improve the robustness of stress representations.
\end{itemize}

\section{Dataset Description}
The study will utilize EEG datasets such as the DEAP dataset, containing EEG signals recorded during emotional states. Data preprocessing will involve noise filtering, feature extraction using wavelet transforms, and normalization for model training.

\section{References}
\begin{thebibliography}{10}
\bibitem{ref1} Author1, A. (2024). \textit{Title of Journal Article}. Journal Name, Volume(Issue), Pages.
\bibitem{ref2} Author2, B. (2024). \textit{Title of Journal Article}. Journal Name, Volume(Issue), Pages.
\bibitem{ref3} Author3, C. (2024). \textit{Title of Journal Article}. Journal Name, Volume(Issue), Pages.
\bibitem{ref4} Author4, D. (2024). \textit{Title of Journal Article}. Journal Name, Volume(Issue), Pages.
\bibitem{ref5} Author5, E. (2024). \textit{Title of Journal Article}. Journal Name, Volume(Issue), Pages.
\bibitem{ref6} Author6, F. (2024). \textit{Title of Journal Article}. Journal Name, Volume(Issue), Pages.
\bibitem{ref7} Author7, G. (2024). \textit{Title of Journal Article}. Journal Name, Volume(Issue), Pages.
\bibitem{ref8} Author8, H. (2024). \textit{Title of Journal Article}. Journal Name, Volume(Issue), Pages.
\bibitem{ref9} Author9, I. (2024). \textit{Title of Journal Article}. Journal Name, Volume(Issue), Pages.
\bibitem{ref10} Author10, J. (2024). \textit{Title of Journal Article}. Journal Name, Volume(Issue), Pages.
\end{thebibliography}

\end{document}
